\section{Schwarm-Topologien}
Zum Bestimmen der sozialen Komponente des Geschwindigkeitsvektors kommen verschiedene Topologien oder Nachbarschaftsbeziehungen zur Anwendung. Diese Beziehungen unter den Partikeln haben einen essentiellen Einfluss auf den Algorithmus. Generell kann zwischen gbest (global best) und lbest (local best) unterschieden werden.

\begin{figure}[htbp]
	\centering
	\begin{minipage}{4cm}
		\centering
		\input{tikz/gbest}
		GBest
	\end{minipage}
	\begin{minipage}{4cm}
		\centering
		\input{tikz/lbest-ring}
		LBest - Ring
	\end{minipage}
	\begin{minipage}{4cm}
		\centering
		\input{tikz/lbest-wheel}
		LBest - Wheel
	\end{minipage}
	\begin{minipage}{4cm}
		\centering
		\input{tikz/lbest-neumann}
		LBest - Von Neumann
	\end{minipage}
	\caption{Schwarm-Topologien}
	\label{schwarm-topologien}
\end{figure}

\subsection{GBest}
Die gbest Topologie geht von einem transparentem Partikelschwarm aus, in welchem die persönlich besten Positionen aller Partikel bekannt sind. Die soziale Komponente des gesamten Schwarms bildet sich also aus dem Abstand zur besten Position des gesamten Schwarms. Ein Vorteil dieser Nachbarschaftsbeziehung ist, dass der Schwarm schnell konvergiert, weil alle Partikel auf die momentan beste Position zulaufen.

\subsection{LBest}
Es existieren einige verschiedene lbest Architekturen. Die verbreitetsten lbest Architekturen sind in Abbildung \ref{schwarm-topologien} dargestellt. Das Grundprinzip von lbest ist, dass jedes Partikel nur auf die besten Positionen seiner Nachbarn zugreifen kann. Wie die Nachbarpartikel bestimmt werden ist von der gewählten Topologie abhängig. Die lbest Methode hat den Vorteil, dass die Wahrscheinlichkeit gegen ein lokales Optimum zu konvergieren bedeutend kleiner ist. Dafür dauert es im Normalfall länger ein Optimum zu finden, als mit der gbest Methode.

\subsection{Wahl der Topologie}
Die Wahl der Topologie hängt stark von der Problemstellung ab. 

\subsection{Subpopulationen}
Um die Vorteile beider Topologien zu vereinen, kann das von genetischen Algorithmen (GA) bekannte Prinzip der Subpopulationen verwendet werden. Dabei wird der eigentliche Schwarm in mehrere Subpopulationen unterteilt, welche unabhängig von einander die PSO ausführen. Oft werden Subpopulationen verwendet, um einzelne Teilgebiete genauer untersuchen zu können.

\newpage

\subsubsection{HS-PSO}
Ein Spezialfall der Partikelschwarmoptimierung ist die hierarchical subpopulation PSO (HS-PSO) \cite{ChuanLin-HSPSO}. Das Prinzip der HS-PSO ist die Unterteilung des Schwarms nach einem hierarchischen Prinzip, wie in einer Unternehmung.

\begin{figure}[htbp]
	\centering
	\input{tikz/hs-pso}
	\caption{Hierarchical Subpopulation PSO}
	\label{hs-pso}
\end{figure}

Die besten Partikel der Subpopulationen \textit{Sub11} - \textit{Sub13} bilden zusammen die Population \textit{Sub1}. Die besten Partikel der Populationen \textit{Sub1} - \textit{Sub2} bilden den Hauptschwarm \textit{Sub}. \\

Die einzelnen Partikelschwärme lassen sich unterschiedlich konfigurieren. So schlägt Chuan Lin vor, Schwärme in tiefen Hierarchiestufen auf das Erforschen des Gesamtgebiets einzustellen (d.h. geringe Sozialkomponente), während die höheren Hierarchiestufen relativ träge sind. Damit lässt sich ein grosses Gebiet erforschen, doch der Schwarm konvergiert kaum gegen lokale Minima. \\

In Benchmark-Tests haben HS-PSO Varianten bedeutend bessere Resultate als herkömmliche gbest und lbest Topologien hervorgebracht. Das Thema ist jedoch erst schwach erforscht. Weitere Arbeiten, vor allem zu adaptiven hierarchischen Strukturen und dem Informationsaustausch zwischen den Subpopulationen sind jedoch angekündigt.

