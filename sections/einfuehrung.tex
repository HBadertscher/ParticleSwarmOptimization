\section{Einführung}
		Die Partikelschwarmoptimierung wurde von James Kennedy \& Russel Eberhart im Jahre 1995 entwickelt. Es handelt sich um ein Nummerischen Optimierungsalgorithmus. Dieser Algorithmus nutzt die Intelligenz von Schwärmen um die Zielfunktion zu Optimieren. Er simuliert das Verhalten von natürlichen Schwärmen um mathematische Problemstellungen zu optimieren.\\
		Es bewegt sich ein Schwarm von Partikel durch den Suchraum. Jedes Partikel hat eine Richtung und Geschwindigkeit sowie eine "Fitness". Die Fitness sagt aus wie gut die momentane Position ist. Partikel mit einer guten Fitness ziehen andere Partikel an, solche mit schlechter Fitness stossen andere Partikel ab. So beeinflussen sich die Partikel gegenseitig und man kann sehr viel Parallel berechnen, was vor allem bei Multicore Computersystemen von Vorteil ist. \\
		Diese Art von Optimierung wurde erst durch die Entwicklung von leistungsstarken Computersystemen möglich. Die ständig steigende Rechenleistung ermöglicht immer umfassendere Simulationen und damit immer umfassendere Optimierungen.
		\subsection{Schwarm}
		In der Natur gibt es viele verschiedene Strukturen und Formen in denen viele einzelne Bestandteile zusammen wirken. Eine davon ist ein Schwarm.\\
		Wenn man von Schwärmen spricht denkt man an Fischschwärme oder an Vogelschwärme die in der freien Wildbahn sich bilden um z.B. als Zugvogelschwarm nach Süden oder Norden zu ziehen oder als Fischschwarm gemeinsam auf Nahrungssuche zu gehen oder sich gegen natürlich Feinde zu verteidigen. Solche Schwärme können das wissen aller einzelnen Nutzen und so Beispielsweise die besten Futterplätze finden.
		\subsection{Schwarmintelligenz}
		Ein Schwarm kann als ganzes eine Eigenschaft haben, die aus den einzelnen Mitglieder des Schwarms nicht voraussagbar ist. So kann eine einzelne Nervenzelle keine Informationen Speicher oder denken. Wenn nun aber Milliarden von Nervenzellen richtig zusammen zu einem Hirn verknüpft sind, kann das ganze denke wie auch Informationen speichern. Dies sind offenbar emergente Eigenschaften, welche nicht durch einzelne Nervenzellen sonder durch ein vielzahl von Nervenzellen zustande kommt.
		\subsubsection{Emergenz}
		Emergente Eigenschaften sind Eigenschaften die durch das Zusammenspiel vom vielen einzelnen Teilen, zum Beispiel in einem Schwarm, entstehen. Durch einzelne Teile in einem Schwarm lässt sicher nicht immer vorher sagen, welche emergente Eigenschaften der Schwarm hat. \\
		Das es ein Zusammenspiel von Bestandteilen ein Verhalten begründet, sagt noch nicht aus, dass dieses Verhalten automatisch Intelligenter als das Verhalten der Bestanteilen ist. In der Natur hat sich aber gezeigt, dass dieses emergente Verhalten oft wesentlich intelligenter ist, als das Verhalten von den einzelnen Bestandteilen.
