\section{Einführung}
		Die Partikelschwarmoptimierung wurde von James Kennedy \& Russel Eberhart im Jahre 1995 entwickelt. Es handelt sich um ein Nummerischen Optimierungsalgorithmus. Dieser Algorithmus nutzt die Intelligenz von Schwärmen um die Zielfunktion zu Optimieren. Er simuliert das Verhalten von natürlichen Schwärmen um mathematische Problemstellungen zu optimieren.\\
		Es bewegt sich ein Schwarm von Partikel durch den Suchraum. Jedes Partikel hat eine Richtung und Geschwindigkeit sowie eine "Fitness". Die Fitness sagt aus wie gut die momentane Position ist. Partikel mit einer guten Fitness ziehen andere Partikel an, solche mit schlechter Fitness stossen andere Partikel ab. So beeinflussen sich die Partikel gegenseitig und man kann sehr viel Parallel berechnen, was vor allem bei Multicore Computersystemen von Vorteil ist. \\
		Diese Art von Optimierung wurde erst durch die Entwicklung von leistungsstarken Computersystemen möglich. Die ständig steigende Rechenleistung ermöglicht immer umfassendere Simulationen und damit immer umfassendere Optimierungen.
		\subsection{Schwarm}
		In der Natur gibt es viele verschiedene Strukturen und Formen in denen viele einzelne Bestandteile zusammen wirken. Eine davon ist ein Schwarm.\\
		Wenn man von Schwärmen spricht denkt man an Fischschwärme oder an Vogelschwärme die in der freien Wildbahn sich bilden um z.B. als Zugvogelschwarm nach Süden oder Norden zu ziehen oder als Fischschwarm gemeinsam auf Nahrungssuche zu gehen oder sich gegen natürlich Feinde zu verteidigen. Solche Schwärme können das wissen aller einzelnen Nutzen und so Beispielsweise die besten Futterplätze finden.
		\subsection{Schwarmintelligenz}
		Ein Schwarm kann als ganzes eine Eigenschaft haben, die aus den einzelnen Mitglieder des Schwarms nicht voraussagbar ist. So kann eine einzelne Nervenzelle keine Informationen Speicher oder denken. Wenn nun aber Milliarden von Nervenzellen richtig zusammen zu einem Hirn verknüpft sind, kann das ganze denke wie auch Informationen speichern. Dies sind offenbar emergente Eigenschaften, welche nicht durch einzelne Nervenzellen sonder durch ein vielzahl von Nervenzellen zustande kommt.
		\subsection{Emergenz}
		Emergente Eigenschaften sind Eigenschaften die durch das Zusammenspiel vom vielen einzelnen Teilen, zum Beispiel in einem Schwarm, entstehen. Durch einzelne Teile in einem Schwarm lässt sicher nicht immer vorher sagen, welche emergente Eigenschaften der Schwarm hat. \\
		Das es ein Zusammenspiel von Bestandteilen ein Verhalten begründet, sagt noch nicht aus, dass dieses Verhalten automatisch Intelligenter als das Verhalten der Bestanteilen ist. In der Natur hat sich aber gezeigt, dass dieses emergente Verhalten oft wesentlich intelligenter ist, als das Verhalten von den einzelnen Bestandteilen. \\
		Die Emergenz ist Grundbestandteil von Algorithmen, welche auf das Verhalten von Schwärmen zurückgreifen.
		
		\subsection{Schwarmalgorithmus}

In den 1980er Jahren begann man das Verhalten von Schwärmen anhand von Computermodellen zu erforschen. Man wollte etwas über Evolution und deren Mechanismen lernen. Aus diesen Simulationen entstand der Schwarmalgorithmus.
In diesem Algorithmus wurden die Naturgesetze mit eingebaut, da man das Verhalten von Tieren in der freien Wildbahn erforschen wollte und diese auch diesen Naturgesetzen unterliegen.

\begin{itemize}
\item Vermeiden von Kollisionen 
\item Angleichen der Geschwindigkeit
\item Angleichen der Flugrichtung
\end{itemize}

Man stellte fest, dass es in solchen Schwärmen nicht möglich war, immer alle anderen Teilnehmer zu beachten, da die Schwärme sonst anfangen zu kreisen. Man fand heraus, das solche Schwärme in der Natur nicht zentral gesteuert werden, sondern es vielmehr ein Zusammenspiel vieler Teile ist, die sich gegenseitig beeinflussen.\\
Mit der Zeit entdeckte man, das solch ein Algorithmus auch zur Lösung von Optimierungsproblemen verwendet werden kann.
Man erkannte auch, dass beim Lösen von mathematischen Optimierungsproblemen die oben erwähnten Naturgesetze keine Rolle mehr spielen. Denn nun geht es nicht mehr um das Verhalten von Tierschwärmen in der freien Natur. Bei mathematischen Problemen ist es z.B. nicht so entscheidend, ob nun 2 Partikel zur selben Zeit am selben Ort sind (was in der Natur zu einer Kollision mit verletzten Tieren führt). So wurde der Schwarmalgorithmus zum Partikelschwarmalgorithmus entwickelt, welcher nicht zum Erforschen von Schwärmen in der Natur gedacht ist, sondern um Optimierungen zu berechnen.
				
		\subsubsection{Algorithmen}
		Es gibt verschiedene Alogrithmen, die die Schwarmintelligenz nutzen. Im Anschluss werden einige kurz vorgestellt. Am weitesten Verbreitet ist der Partikel Schwarmalgorithmus sowie der Ameisenalgorithmus, jedoch gibt es noch weiter Algorithmen, welche ebenfalls erforscht und weiterentwickelt werden. Auf diese Algorithmen wird im weitern nicht weiter eingegangen.
		
		\paragraph{Ant Colony Alogrithm}
		$\;$ \\
		Auch Ameisen nutzen die Intelligenz des Ganzen, wenn sie Futterplätze nutzen. Hat eine Ameise einen Futterplatz gefunden, hinterlässt sie auf dem Rückweg Pheromone, an denen sich andere Ameisen orientieren und so den Futterplatz und jeweils den Rückweg finden. Findet nun eine Ameise einen schnelleren Weg zurück so verflüchtigen sich diese Pheromone weniger schnell, da es weniger Zeit braucht für einen Weg. Nun werden die andere Ameisen aufmerksam auf diese stärkere Spur und schlussendlich nutzen alle diesen schnelleren Weg. Dieses Prinzip wird im Ameisenalgorithmus (ant colony optimization algorithm) ausgenützt.\\Dieser Algorithmus wird typischerweise für eine Pfadoptimierung verwendet.
		
		\paragraph{Invasive Weed}
		$\;$ \\
		Dieser Algorithmus orientiert an der Fortpflanzung von Pflanzen. Es werden Pflanzen frei über das Suchgebiet verteilt und dann die Fitness bewertet. Je Fitter eine Pflanze ist desto mehr Pflanzensamen verteilt sie. Auch bei diesen Pflanzen wird die Fitness bewertet. Das ganze wird solange fortgeführt, bis eine maximal Anzahl Pflanzen, die zuvor definiert wurde, erreicht ist und dann wird die Fitness aller Pflanzen bewertet und nur die der besten werden in einen neuen Durchgang mitgenommen. Dies wird solange wiederholt, bis das Abbruchkriterium erfüllt ist.
				
		
		\paragraph{Bees Algorithm}
		$\;$ \\
		Beim Bienenalgorithmus hat man, wie der Name schon sagt, bei den Bienen abgeschaut.\\
		In einem ersten Schritt werden "Scouts" frei über das Suchgebiet verteilt und im Anschluss deren Fitness angeschaut. Bei dem "Scout" mit der besten Fitness werden in der Umgebung erneut "Bienen" verteilt und deren Fitness beurteilt. Dieser Schritt wiederholt sich, bis man ein Ergebnis erreicht hat, das das Abbruchkriterium erfüllt.\\Dieser Algorithmus wird typischerweise für Kombinatorische oder funktionale Optimierung verwendet.
		
		\paragraph{Firefly Algorithm} 
		$\;$ \\
		Der Firefly Algorithm nutzt das Paarungsverhalten von Glühwürmchen. Glühwürmchen werden Unisexuel implementiert damit jedes Tier für das andere als Partner infrage kommt. \\
		Zuerst werden Glühwürmchen frei auf dem Suchgebiet verteilt. Danach wird die Fitness beurteilt. Je fitter ein Glühwürmchen ist, desto heller leuchtet es und je heller es leuchtet desto interessanter ist ein Glühwürmchen für die Artgenossen. Das heisst, jedes einzelne bewegt sich auf das Tier zu, das aus seiner Sicht am hellsten ist. Je näher man sich auf ein Glühwürmchen zu bewegt, desto heller leuchtet es. So werden bald alle Glühwürmchen Richtung Optimum wandern. Würden nun alle gleich hell leuchten, sprich dieselbe Fitness haben, würden sich die Tiere zufällig im Suchraum bewegen.
		
		