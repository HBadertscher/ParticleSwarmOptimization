\section{Schwarmalgorithmus}

In den 1980 Jahren begann man das Verhalten von Schwärmen zu erforschen. Man simulierte Schwärme an Computermodellen. Man wollte etwas über Evolution und deren Mechanismen lernen. Aus diesen Simulationen entstand der Schwarmalgorithmus.
Bei diesen Simulationen wurden die Naturgesetze mit eingebaut, da man ja das Verhalten von Tieren in der freien Wildbahn erforschen wollte und diese auch diesen Naturgesetzen unterliegen.
\begin{itemize}
\item Vermeiden von Kollisionen 
\item Angleichen der Geschwindigkeit
\item Angleichen der Flugrichtung
\end{itemize}

Weiter musste man feststellen, dass es in solchen Schwärmen nicht möglich war, immer alle andere Teilnehmer zu beachten, da er sonst anfängt zu Kreisen. Man fand heraus, das solche Schwärme in der Natur nicht zentral gesteuert werden, sonder vielmehr ein Zusammenspiel vieler Teile ist, die sich gegenseitig Beeinflussen.\\
Mit der Zeit entdeckte man, das solch ein Algorithmus auch zur Lösung von Optimierungsproblemen verwendet werden kann.
Man erkannte auch, das beim Lösen von Mathematischen Optimierungsproblemen die oben erwähnten Naturgesetze nicht mehr so eine Rolle spielen. Denn nun geht es nicht mehr um das Verhalten von Tierschwärmen in der freien Natur. Bei mathematischen Problemen ist es z.B. nicht so entscheidend ob nun 2 Partikel zur selben Zeit am selben Ort sind, was in der Natur zu einer Kollision mit verletzten Tieren führt. So wurde der Schwarmalgorithmus zum Partikelschwarmalgorithmus entwickelt, welcher nicht zum Erforschen von Schwärmen in der Natur gedacht ist, sondern um Optimierungen zu berechnen.