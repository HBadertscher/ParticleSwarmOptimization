\subsection{Schwarmalgorithmus}

In den 1980er Jahren begann man das Verhalten von Schwärmen anhand von Computermodellen zu erforschen. Man wollte etwas über Evolution und deren Mechanismen lernen. Aus diesen Simulationen entstand der Schwarmalgorithmus.
In diesem Algorithmus wurden die Naturgesetze mit eingebaut, da man das Verhalten von Tieren in der freien Wildbahn erforschen wollte und diese auch diesen Naturgesetzen unterliegen.

\begin{itemize}
\item Vermeiden von Kollisionen 
\item Angleichen der Geschwindigkeit
\item Angleichen der Flugrichtung
\end{itemize}

Man stellte fest, dass es in solchen Schwärmen nicht möglich war, immer alle anderen Teilnehmer zu beachten, da er sonst anfängt zu kreisen. Man fand heraus, das solche Schwärme in der Natur nicht zentral gesteuert werden, sondern es vielmehr ein Zusammenspiel vieler Teile ist, die sich gegenseitig beeinflussen.\\
Mit der Zeit entdeckte man, das solch ein Algorithmus auch zur Lösung von Optimierungsproblemen verwendet werden kann.
Man erkannte auch, das beim Lösen von mathematischen Optimierungsproblemen die oben erwähnten Naturgesetze keine Rolle mehr spielen. Denn nun geht es nicht mehr um das Verhalten von Tierschwärmen in der freien Natur. Bei mathematischen Problemen ist es z.B. nicht so entscheidend, ob nun 2 Partikel zur selben Zeit am selben Ort sind (was in der Natur zu einer Kollision mit verletzten Tieren führt). So wurde der Schwarmalgorithmus zum Partikelschwarmalgorithmus entwickelt, welcher nicht zum Erforschen von Schwärmen in der Natur gedacht ist, sondern um Optimierungen zu berechnen.