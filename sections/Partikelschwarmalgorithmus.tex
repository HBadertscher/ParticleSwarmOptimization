\section{Partikelschwarm-Algorithmus}

\subsection{Algorithmus}
Für die Partikelschwarm-Optimierung sind folgende Zustandsdaten zu jedem Partikel notwendig: \\
\begin{tabular}{ll}
$x_i$: & aktuelle Position\\
$v_i$: & aktuelle Geschwindigkeit\\
$p_i$: & Persönlich beste Position\\
\end{tabular} \\

\subsubsection{Initialisierung}
Üblich ist die Initialisierung gemäss folgenden Formeln: \\
\begin{align}
	x_i(0) &= U(min,max) \\
	v_i(0) &= \frac{U(min,max) - x_i(0)}{2} \label{Vi-old} \\ 
	p_i(0) &= x_i(0)
\end{align} \\
Die Erfahrung hat gezeigt, dass mit der Initialisierung der Geschwindigkeit gemäss Gleichung \ref{Vi-old} die Partikel bei hohen Dimensionen den Suchraum praktisch unmittelbar verlassen. Um dies zu beheben wurde folgende angepasste Initialisierung eingeführt: \\
\begin{equation}
	v_i(0) = U(min - x_i(0), max - x_i(0))
\end{equation}

\subsubsection{Update der Geschwindigkeit}
Bei jeder Iteration werden die Zustandsdaten für jedes Partikel neu berechnet. Die neue Geschwindigkeit ist eine Linearkombination von drei Vektoren. Die Position wird gemäss Gleichung \ref{Pos-Update} aktualisiert. \\
\begin{align}
	v_{i}(t+1) &= \mathcal{C}(v_i(t),\, p_i(t)-x_i(t),\, l_i(t)-x_i(t)) \\
	x_{i}(t+1) &= x_i(t) + v_i(t+1) \label{Pos-Update}
\end{align}

Über die korrekte Wahl der Linearkombination $\mathcal{C}$ wurden bereits ganze Arbeiten geschrieben. Weit verbreitet, wenn auch nicht ideal ist die Berechnung gemäss Gleichung \ref{Vel-Update}. \\
\begin{equation}
	v_i(t+1) = w v_i(t) + U(0,c) (p_i(t)-x_i(t)) + U(0,c) (l_i(t)-x_i(t))\label{Vel-Update}
\end{equation}
Die Parameter werden gemäss \cite{Clerc-Stagnation} folgendermassen gewählt:
\begin{equation}
	\left\lbrace \begin{array}{lllll}
		w & = & \frac{1}{2 \ln(2)} & \simeq & 0.721 \\
		c & = & \frac{1}{2} + \ln(2) & \simeq & 1.139 \\
	\end{array}	\right. 
\end{equation} \\

Die Herleitung der neuen Geschwindigkeit lässt sich mittels virtuellen Punkten übersichtlich visualisieren. Die Punkte $x'_i$ und $x''_i$ werden zufällig aus den zu den Achsen parallelen, gelb bzw. grün hinterlegten Bereichen generiert.  \\
\begin{figure}[htbp]
	\centering
	\input{tikz/new-velocity}
	\caption{Visualisierung der Geschwindigkeit}
	\label{Fig-Visualisierung-Geschwindigkeit}
\end{figure}


\subsection{Grösse des Schwarms}
Über die ideale Grösse des Partikelschwarms lassen sich keine exakten Angaben machen. Gemäss \cite{Clerc-Standards} lässt sich die Grösse $S$ des Schwarms in Abhängigkeit der Dimension $D$ wie folgt ausdrücken.
\begin{equation}
	S = 10 + \left[ 2 \cdot \sqrt{D} \right]
\end{equation}
Die Näherung führt jedoch oft zu ungeeigneten Werten, weshalb Maurice Clerc vorschlägt, einen beliebigen Wert um $40$ zu wählen. Die theoretischen Grundlagen zur idealen Grösse des Schwarms sind nicht bekannt.

\subsection{Parallelisierung}
Aus der Beschreibung des Algorithmus wird schnell ersichtlich, dass die Partikelschwarm-Optimierung sehr gut für eine parallelisierte Berechnung geeignet ist. Die Aktualisierung der Geschwindigkeit und Position kann für alle Partikel gleichzeitig, parallel erfolgen. Bei rechenintensiven Problemen ist dies ein enormer Vorteil gegenüber anderen Optimierungsmethoden.