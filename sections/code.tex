\section{Code}
Der komplette C++ Code befindet sich im Anhang. In diesem Teil wird der Code beschrieben.

\subsection{Position-Klasse}
Die Klasse \textit{Position} stellt einen mehrdimensionalen Datentyp zur Verfügung. 
Damit ist es möglich, Probleme jeder beliebigen Dimension zu lösen.

\subsection{Particle-Klasse}
In der \textit{Particle}-Klasse werden die Positionen und Geschwindigkeiten jedes Partikels
als Attribute gespeichert. Weiter werden Funktionen für Geschwindigkeits- und Positionsupdates
zur Verfügung gestellt.

\subsection{Swarm-Klasse}
Die Klasse \textit{Swarm} dient dazu, einen Schwarm von Partikeln zu erzeugen und zu initialisieren.
Zusätzlich wird eine Funktion zur Ausführung der Optimierung bereitgestellt.

\subsection{Random}
Die selbst erstellte Funktion \textit{getRand} dient dazu, Zufallszahlen im Bereich zwischen \textit{min} und \textit{max} zu erzeugen. Diese werden in der Initialisierung und bei jeder Iteration des Schwarms benötigt. 