\section{Einschränkung des Zielgebiets}
Häufig kommt es vor, dass das Optimum innerhalb eines bestimmten Zielgebiets gesucht wird. Daher müssen Mittel gefunden werden, um zu verhindern dass die Partikel einen Definitionsbereich verlassen und nicht mehr zur Lösungsfindung beitragen. Die einfachste Möglichkeit ist, den Fitnesswert von Partikeln ausserhalb des Definitionsgebiets auf einen schlechten Wert zu setzen. Trotzdem benötigt das Partikel meist mehrere Iterationen um zurück in das Definitionsgebiet zu finden.

Eine verbreitete Lösung ist, das Partikel auf die Grenze des Bereichs und die Geschwindigkeit auf 0 zu setzen. Dieser Ansatz hat sich gegenüber dem ``Abprallen'' an der Bereichsgrenze durchgesetzt, weil so Optima am Rand des Zielgebiets besser erkannt werden können.

\section{Maximale Geschwindigkeit}
Ein bekanntes Problem der Partikelschwarm-Optimierung ist, dass die Geschwindigkeit eines Partikels schnell sehr hohe Werte annimmt. Das kann so weit gehen, dass ganze Teilgebiete durch die hohe Geschwindigkeit übersprungen werden. Die Lösung für dieses Problem ist die Einführung einer maximalen Geschwindigkeit, \textit{velocity clamping} genannt. Die Geschwindigkeit wird also nach folgender Formel berechnet:
\begin{equation}
	v_{ij}(t+1) = 
	\begin{cases}
		v_{ij}(t+1) & \text{if} |v_{ij}(t+1)| < V_{max,j} \\
		V_{max,j} & \text{if} |v_{ij}(t+1)| \geq V_{max,j}
	\end{cases}
	\label{eqn-max-velocity}
\end{equation}
Durch diese Erweiterung sorgt dafür, dass ein grösserer Bereich des Suchraums erforscht wird. Der optimale Wert für $V_{max,j}$ ist vom Problem abhängig und kann nicht generell bestimmt werden. \\

Wie die Gleichung \ref{eqn-max-velocity} zeigt, wird die Geschwindigkeit für jede Dimension einzeln beschränkt. Damit ergibt sich graphisch folgende Situation:

\begin{figure}[htbp]
	\centering
	\includegraphics{tikz/max-velocity.tex}
	\caption{Einführen einer maximalen Geschwindigkeit}
	\label{fig-max-velocity}
\end{figure}

Diese Geschwindigkeitsbegrenzung führt jedoch zu Problemen, wenn alle Geschwindigkeiten höher als die Begrenzung sind: Die Teilchen suchen dann nur an den Kanten eines $n$-dimensionalen Hyperwürfels. Als Gegenmassnahme kann $V_{max,j}$ dynamisch angepasst werden, wenn sich die global beste Position gbest über eine Anzahl $\tau$ Iterationen nicht verbessert:
\begin{equation}
	V_{max,j} = 
	\begin{cases}
		\beta V_{max,j} & \text{if} f(\hat{y}(t)) \geq f(\hat{y}(t-t')) \\
			& \forall \: t' = \{ 1, ..., \tau \} \\
		V_{max,j} & \text{otherwise}
	\end{cases}
\end{equation}

Um eine höhere Genauigkeit zu erreichen kann die maximale Geschwindigkeit im Verlauf der Zeit exponentiell verringert werden:
\begin{equation}
	V_{max,j}(t+1) = \left(1-\left(\frac{t}{T}\right)^{\alpha}\right) V_{max,j}(t)
\end{equation}
$T$ ist hier die maximale Anzahl Iterationen. Der Faktor $\alpha$ ist abhängig von der Problemstellung und es existieren gemäss \cite{Han-Modification} keine mathematischen Hintergründe zur exakten Bestimmung. \\